\documentclass[12pt, a4paper]{article}

\usepackage{lipsum}

\usepackage{indentfirst}
\usepackage{parskip}
\usepackage{setspace}
\setlength{\parskip}{1.6em}
\setlength{\parindent}{1.25cm}
\doublespacing


\usepackage[left=1in, right=1in, top=1in, bottom=1in]{geometry}
\usepackage{fancyhdr}
\pagestyle{fancy}
\setlength{\headheight}{15pt}
\renewcommand{\headrulewidth}{0pt}
\fancyhf{}
\fancyhead[R]{\thepage}



\usepackage{amsmath}
\usepackage[dvipsnames]{xcolor}
\usepackage{mathtools}
\usepackage{amsfonts}
\usepackage{titlesec}
\usepackage{blindtext}


\usepackage{graphicx}
\graphicspath{ {./images/} }
\usepackage{wrapfig}
\usepackage{float}


\usepackage{multirow}
\usepackage{array}
\usepackage{tabu}
\titleformat{\section}
{\normalfont\large\bfseries}{\thesection}{1em}{}
\titleformat{\subsection}
{\normalfont\large\bfseries}{\thesubsection}{1em}{}


\counterwithin{equation}{section}

\usepackage{hyperref}
\urlstyle{same}



\title{
    IB Mathematics AA Higher Level\\
    
    Exploring and implementing 3D graphics through linear algebra
}
\author{Lars Sawadsky}
\date{\today}
\begin{document}

    \begin{titlepage}
        \begin{center}
            \vspace*{9cm}

            IB Mathematics AA Higher Level\\
            \medskip
            Exploring and implementing 3D computer graphics using linear algebra

            \vspace{8cm}

            L. Sawadsky\\
            Internal Assessment: Mathematics AA HL\\
            Session: May 2026\\
            Word Count:
        \end{center}
        \newpage
    \end{titlepage}

    \tableofcontents
    \newpage

    \section*{Introduction}
    \addcontentsline{toc}{section}{Introduction}
    
    Representing our three-dimensional world in the two dimensions
    our computer screens allow us has proven not to be a simple task,
    but thankfully one that centuries of research in linear algebra
    happen to assist greatly in. Simulating a three-dimensional 
    environment in a computer system is done largely using research
    in physics: laws and constants that govern the behaviour of rigid
    bodies, fluid, and light are replicated in programs to create a
    world that functions, more or less, like ours. However, there is 
    a problem designers run into that has no parallel in our world. 
    We understand the laws by which physical things change, but \textit{representing} 
    these changes is a challenge unique to computerized versions of
    the world. This investigation will explore two aspects of this non-physical 
    issue: the projection of a set of given vertices in 3D space 
    onto a 2D screen, and some
    common manipulations of those vertices: translations, stretches, 
    and rotations. 

    \section*{Projection of given points}
    \addcontentsline{toc}{section}{Projection of given points}

    The first challenge this investigation will tackle is the visualization of a given set of points in \mathbf{R}

    


\end{document}